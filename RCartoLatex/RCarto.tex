\documentclass[a4paper]{article}\usepackage[]{graphicx}\usepackage[]{color}
%% maxwidth is the original width if it is less than linewidth
%% otherwise use linewidth (to make sure the graphics do not exceed the margin)
\makeatletter
\def\maxwidth{ %
  \ifdim\Gin@nat@width>\linewidth
    \linewidth
  \else
    \Gin@nat@width
  \fi
}
\makeatother

\definecolor{fgcolor}{rgb}{0.345, 0.345, 0.345}
\newcommand{\hlnum}[1]{\textcolor[rgb]{0.686,0.059,0.569}{#1}}%
\newcommand{\hlstr}[1]{\textcolor[rgb]{0.192,0.494,0.8}{#1}}%
\newcommand{\hlcom}[1]{\textcolor[rgb]{0.678,0.584,0.686}{\textit{#1}}}%
\newcommand{\hlopt}[1]{\textcolor[rgb]{0,0,0}{#1}}%
\newcommand{\hlstd}[1]{\textcolor[rgb]{0.345,0.345,0.345}{#1}}%
\newcommand{\hlkwa}[1]{\textcolor[rgb]{0.161,0.373,0.58}{\textbf{#1}}}%
\newcommand{\hlkwb}[1]{\textcolor[rgb]{0.69,0.353,0.396}{#1}}%
\newcommand{\hlkwc}[1]{\textcolor[rgb]{0.333,0.667,0.333}{#1}}%
\newcommand{\hlkwd}[1]{\textcolor[rgb]{0.737,0.353,0.396}{\textbf{#1}}}%

\usepackage{framed}
\makeatletter
\newenvironment{kframe}{%
 \def\at@end@of@kframe{}%
 \ifinner\ifhmode%
  \def\at@end@of@kframe{\end{minipage}}%
  \begin{minipage}{\columnwidth}%
 \fi\fi%
 \def\FrameCommand##1{\hskip\@totalleftmargin \hskip-\fboxsep
 \colorbox{shadecolor}{##1}\hskip-\fboxsep
     % There is no \\@totalrightmargin, so:
     \hskip-\linewidth \hskip-\@totalleftmargin \hskip\columnwidth}%
 \MakeFramed {\advance\hsize-\width
   \@totalleftmargin\z@ \linewidth\hsize
   \@setminipage}}%
 {\par\unskip\endMakeFramed%
 \at@end@of@kframe}
\makeatother

\definecolor{shadecolor}{rgb}{.97, .97, .97}
\definecolor{messagecolor}{rgb}{0, 0, 0}
\definecolor{warningcolor}{rgb}{1, 0, 1}
\definecolor{errorcolor}{rgb}{1, 0, 0}
\newenvironment{knitrout}{}{} % an empty environment to be redefined in TeX

\usepackage{alltt}
\usepackage[francais]{babel}
\usepackage[T1]{fontenc}
\usepackage[utf8]{inputenc} % le texte source doit être utf8 sinon latin1
\IfFileExists{upquote.sty}{\usepackage{upquote}}{}
\begin{document}

\title{Project Report Template}
\author{JcB \& Graham Williams}
\maketitle\thispagestyle{empty}

\section{Introduction}
Un paragraphe ou deux pour introduire le sujet.

\section{Question de recherche}
Describe discussions with client (business experts) and record
decisions made and shared understanding of the business problem.

\section{Les données source}
Identifier les les sources de données et discuter de leur accès avec les propriétaires de ces données. Documenter les données sources, leur intégrité, leur origine et les dater.

\section{Préparation des données}

charger les données dans R et réaliser diverses transformations pour les adapter à l'analyse et à la modélisation.

\section{Exploration des données}
Il faut toujours commencer par comprendre le données en les explorant sous différents angles.
Les résumés et des graphiques simples sont une aide précieuse.

\section{Construire un modèle}
Inclure tous les modèles imaginés et les différents paramètres testés.
Inclure le code R et les évaluations des modèles

\section{Deployer}
Choisir le modèle à déployer et l'exporter.


\section{Utilisation de RStudio}

RStudio supporte le style ancien \emph{Sweave} et la version moderne \emph{knir}. Pour utiliser \emph{knir}, il faut en informer RStudio.
Dans \texttt{Tools} $\longrightarrow$ \texttt{Project option}, sélectionner l'icone \textbf{Sweave}, puis l'option \textbf{knitr} qui sera utilisé pour weaver les fichiers \emph{.Rnw}.

\section{Insérer du code R}

\begin{knitrout}
\definecolor{shadecolor}{rgb}{0.969, 0.969, 0.969}\color{fgcolor}\begin{kframe}
\begin{alltt}
\hlstd{x} \hlkwb{<-} \hlkwd{runif}\hlstd{(}\hlnum{1000}\hlstd{)} \hlopt{*} \hlnum{1000}
\hlkwd{head}\hlstd{(x)}
\end{alltt}
\begin{verbatim}
## [1] 668.2 224.5 206.1 814.7 173.3 157.8
\end{verbatim}
\begin{alltt}
\hlkwd{mean}\hlstd{(x)}
\end{alltt}
\begin{verbatim}
## [1] 496
\end{verbatim}
\end{kframe}
\end{knitrout}

Pour afficher sur une ligne utiliser \begin{verbatim} \Sexpr{} \end{verbatim}, par exemple \begin{verbatim} 2014-09-14 \end{verbatim} donne 2014-09-14.


\end{document}
